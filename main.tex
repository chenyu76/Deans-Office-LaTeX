\documentclass[12pt,a4paper]{ctexart}

% 配置
\newlength{\PageMargin}
\setlength{\PageMargin}{1in}      % 全局外框边距(距离纸张边缘的距离)

\newlength{\ContentGap}
\setlength{\ContentGap}{0.1in}    % 外框与页面内容之间的距离距

\newlength{\ContentMargin}
\setlength{\ContentMargin}{\dimexpr\PageMargin + \ContentGap\relax}

\newlength{\RowHeight}
\setlength{\RowHeight}{1.2em}     % 默认行高(最小高度)

\def\GlobalLineWidth{0.5pt}            % 外框线宽
\def\InnerLineWidth{\GlobalLineWidth}  % 内部表格线宽(目前使用相同宽度)

\usepackage[
  margin=\ContentMargin,
  top=\ContentMargin,
  bottom=\ContentMargin,
  nomarginpar
]{geometry}
\usepackage{tikz}
\usepackage{xltabular} % 用于跨页长表格
\usepackage{tabularx}  % 用于单行多列计算
\usepackage{atbegshi}  % 用于每页画框
\usepackage{calc}      % 用于长度计算
\usepackage{environ}   % 用于更好的环境定义
\usepackage{lipsum}    % 插入示例内容

% 计算表格内容的总宽度 (版心宽度)
\newlength{\ContentWidth}
\setlength{\ContentWidth}{\dimexpr\textwidth+\ContentGap+\ContentGap\relax}

% 定义开关
\newif\ifShowPageFrame \ShowPageFramefalse

% 绘制全页外框 (Hook到每页输出)
\newcommand{\DrawFrameOverlay}{%
  \ifShowPageFrame
  \begin{tikzpicture}[overlay, remember picture]
    % 计算坐标:左上角 (L, T) 和 右下角 (R, B)
    % Geometry的坐标原点和TikZ current page有所不同,这里直接用current page偏移计算最稳健
    \draw[line width=\GlobalLineWidth]
    ([xshift=\PageMargin, yshift=-\PageMargin]current page.north west)
    rectangle
    ([xshift=-\PageMargin, yshift=\PageMargin]current page.south east);

    % 如果需要调试,可以打开下面的网格
    % \draw[help lines, step=1cm] (current page.north west) grid
    % (current page.south east);
  \end{tikzpicture}
  \fi
}
\AtBeginShipout{\AtBeginShipoutAddToBox{\DrawFrameOverlay}}

% 通用组件命令

% 环境:开启大表模式
\newenvironment{TablePage}{%
  \clearpage
  \ShowPageFrametrue
  \setlength{\parindent}{0pt} % 消除缩进
  \setlength{\parskip}{0pt}   % 消除段间距
}{%
  \clearpage
  \ShowPageFramefalse
}

% 命令:绘制横向分割线
\newcommand{\SepLine}{%
  \par
  \nointerlineskip % 防止垂直间距干扰
  \noindent
  \begin{tikzpicture}[overlay, remember picture]
    % \pgfmathsetmacro{\h}{-1};
    \pgfmathsetmacro{\m}{\PageMargin};
    % \path (current page.center);
    % \pgfgetlastxy{\centerx}{\centery}; % 提取 x 和 y 坐标
    \path (current page.west);
    \pgfgetlastxy{\westx}{\westy};
    \path (current page.east);
    \pgfgetlastxy{\eastx}{\easty};
    \draw ({\westx + \m}, 0) -- ({\eastx - \m}, 0);
  \end{tikzpicture}
  \par
  \nointerlineskip
}

% 绘制带有垂直间距的分隔线
\newcommand{\vspaceSepLine}{
  \vspace{0.5em}

  \SepLine

  \vspace{0.5em}
}

% 命令:多列行
% 用法: \TableRow[最小高度]{列格式}{内容1 & 内容2...}
% 示例: \TableRow{X|c|X}{ 左边 & 中间 & 右边 }
% 注意:列格式不需要写两边的竖线,系统会自动处理
\NewDocumentCommand{\TableRow}{O{\RowHeight} m m}{%
  \noindent
  % 使用 tabularx 占满行宽
  % @{} 消除左右默认间距,使表格紧贴版心边缘
  \begin{tabularx}{\textwidth}{@{} #2 @{}}
    % 在每个单元格内插入一个不可见的rule撑开高度
    \rule{0pt}{#1} #3
  \end{tabularx}%
  \par
  \nointerlineskip % 紧接着画线或下一行
  \SepLine % 每行结束后自动画分割线
}

% 命令:跨页长文本块
% 类似于 Word 的大文本框,可以确定高度
% 用法: \begin{LongTextField}[高度] ... \end{LongTextField}
\NewEnviron{LongTextField}[1][5cm]{%
  \noindent
  \begin{minipage}[t][#1][t]{\textwidth}
    \vspace{0.5em} % 顶部留白
    \BODY
  \end{minipage}
  \par
  \nointerlineskip
  \SepLine
}

% 用于遮住本身已有的框线的一段额外距离
\newlength{\smallOverlayLength}
\setlength{\smallOverlayLength}{0.5cm}

% 命令:自定义表头 (覆盖上边框)
% 用法: \PageTitle{标题}{标题高度}
\newcommand{\PageTitle}[2]{%
  \par
  \nointerlineskip
  \noindent
  \newlength{\HeadingHeight}
  \setlength{\HeadingHeight}{#2}
  \begin{tikzpicture}[overlay, remember picture]
    % 用白色矩形遮盖顶部的全局边框线
    \fill[white]
    ([xshift=\PageMargin+\GlobalLineWidth-\smallOverlayLength,
    yshift=-\PageMargin+\smallOverlayLength]current page.north west)
    rectangle
    ([xshift=-\PageMargin-\GlobalLineWidth+\smallOverlayLength,
    yshift=-\PageMargin-\HeadingHeight]current page.north east);

    % 绘制标题内容
    \node[yshift=-1.2cm] at ([xshift=0.5\paperwidth,
    yshift=-\PageMargin]current page.north west) {
      \begin{minipage}{\ContentWidth}
        #1
      \end{minipage}
    };

    % 补画标题下方的横线
    \draw[line width=\GlobalLineWidth]
    ([xshift=\PageMargin, yshift=-\PageMargin-\HeadingHeight]current
    page.north west) --
    ([xshift=-\PageMargin, yshift=-\PageMargin-\HeadingHeight]current
    page.north east);
  \end{tikzpicture}
  \vspace{\dimexpr\HeadingHeight-\ContentGap-1em\relax}
  \par
  \nointerlineskip
}

% 命令:提前结束表格
\newcommand{\EndTableEarly}{
  \BlankBetweenTable{100cm}
}

% 命令:创建一段表格之间的空白
% 参数为高度
\newcommand{\BlankBetweenTable}[1]{
  \par
  \nointerlineskip
  \noindent
  \begin{tikzpicture}[overlay, remember picture]
    \pgfmathsetlengthmacro{\h}{-(#1)}
    \pgfmathsetmacro{\m}{\PageMargin};
    \pgfmathsetmacro{\l}{\smallOverlayLength};
    \path (current page.west);
    \pgfgetlastxy{\westx}{\westy};
    \path (current page.east);
    \pgfgetlastxy{\eastx}{\easty};
    \fill[white]({\westx + \m - \l}, 0) rectangle ({\eastx - \m + \l}, \h);
    \draw ({\westx + \m}, 0) -- ({\eastx - \m}, 0);
    \draw ({\westx + \m}, \h) -- ({\eastx - \m}, \h);
  \end{tikzpicture}
  \par
  \nointerlineskip
}


\begin{document}
% 开始带框页面
\begin{TablePage}
  % 生成表头
  \PageTitle{
    \centering
    \huge \bfseries 本科毕业设计(论文)开题报告 \\[0.5em]
    \large XX大学 · 2077届
  }{
    2.5cm % 标题高度
  }
  % 单行多列信息
  % 语法:\TableRow{列格式}{内容}
  % X 代表自动宽度,c 代表居中,| 代表竖线
  % 技巧:通过调整 X 的系数 (如 >{\hsize=0.5\hsize}X ) 可以控制列宽比例
  \TableRow{c|X}{
    \textbf{题\quad\quad 目} & 基于LaTeX的通用表格模板设计与实现
  }

  \TableRow{ c | X | c | X }{
    \textbf{姓\quad\quad 名} & 张三 & \textbf{学\quad 号} & 2073123456
  }

  \TableRow{ c | X | c | X }{
    \textbf{学\quad\quad 院} & 计算机与软件学院 & \textbf{专\quad 业} & 计算机科学
  }

  \TableRow{ c | X | c | X }{
    \textbf{指导教师} & 王教授 & \textbf{职\quad 称} & 教授
  }

  % 标题栏,l 表示左对齐
  \TableRow{l}{ \textbf{一、本选题的意义及国内外发展状况:} }

  % 大段文本内容
  % 如果内容很少,可以用 LongTextField 指定高度占位
  \begin{LongTextField}[8cm]
    样式修改后需要第二次编译边框线才能正确显示。

    这个区域的高度被强制设为 8cm,适合手写填表。

    如果需要跨页的长文本,请看下一节。

  \end{LongTextField}

  \TableRow{l}{ \textbf{二、研究内容:} }

  % 对于正文非常长的部分,不要用 \TableRow,直接写内容,
  % 侧边的框由 overlay 提供,底部的分割线由下一行 \SepLine 提供。
  \vspace{0.5em}
  这里是研究内容的正文。由于我们使用的是 overlay 画外框,这里的文本可以像普通 LaTeX 文档一样编写。

  可以随意使用列表:
  \begin{itemize}
    \item 这是一个列表项
    \item 这是另一个列表项
  \end{itemize}

  也可以显示公式:
  \begin{equation}
    e^{\pi i} + 1 = 0
  \end{equation}

  通过下面这个命令插入一大堆内容
  \lipsum[8-15]

  \vspace{0.5em} % 像这个增加垂直间距的命令可能需要多次使用

  \SepLine % 结束这一大块内容时,手动画一条分割线

  % 结尾签字栏
  \TableRow{X | X}{
    \centering \textbf{指导教师意见} & \centering \textbf{学院审核意见}
  }

  % 最后一栏,高度设大一点
  \TableRow[4cm]{X | X}{
    \vspace{2.5cm} & \vspace{2.5cm}
  }

  \vspace{0.5em}

  \lipsum[1-1]

  下面这个命令可以创建这样一条上下边距较为合适的分隔线

  \vspaceSepLine

  上面这个命令可以创建这样一条上下边距较为合适的分隔线

  \vspace{0.5em}

  \BlankBetweenTable{3.5em}

  \vspace{0.5em}

  你可以使用上面的BlankBetweenTable命令创建一段空白,参数是空白的高度

  这个高度需要自己摸索摸索控制有多高。

  \vspace{0.5em}

  \lipsum[1-1]

  % 如果需要参考文献
  % \bibliographystyle{plain}  % 选择参考文献的格式
  % \bibliography{references.bib}  % 引用.bib文件
  % 可以在导言区加入下面两行命令删除“参考文献”标题
  % \usepackage{natbib}
  % \renewcommand{\bibsection}{}

  EndTableEarly命令会提前结束表格,如果不使用,表格会延伸到本页结束。

  \vspace{0.5em}

  \EndTableEarly

  \vspace{0.5em}

  你也可以在这个表格外面写点东西,但是如果触发换页,只要在TablePage环境中,仍会有页面外框。

\end{TablePage}

TablePage环境外面的内容可以不带框。但是需要强制换页。

\end{document}
